\documentclass[a4paper]{ltjsarticle}
\usepackage[top=25truemm,bottom=30truemm,left=20truemm,right=20truemm]{geometry}
\usepackage{hinodeji}

\begin{document}
対応表(最後に あ を加えている)\par
あいうえお: \hinodeji[false]{あいうえおあ}\par
かきくけこ: \hinodeji[false]{かきくけこあ}\par
さしすせそ: \hinodeji[false]{さしすせそあ}\par
たちつてと: \hinodeji[false]{たちつてとあ}\par
なにぬねの: \hinodeji[false]{なにぬねのあ}\par
はひふへほ: \hinodeji[false]{はひふへほあ}\par
まみむめも: \hinodeji[false]{まみむめもあ}\par
やゆよ	: \hinodeji[false]{やゆよあ}\par
らりるれろ: \hinodeji[false]{らりるれろあ}\par
わゐゑをん: \hinodeji[false]{わゐゑをんあ}\par
がぎぐげご: \hinodeji[false]{がぎぐげごあ}\par
ざじずぜぞ: \hinodeji[false]{ざじずぜぞあ}\par
だぢづでど: \hinodeji[false]{だぢづでどあ}\par
ばびぶべぼ: \hinodeji[false]{ばびぶべぼあ}\par
ぱぴぷぺぽ: \hinodeji[false]{ぱぴぷぺぽあ}\par
ぁぃぅぇぉ: \hinodeji[false]{ぁぃぅぇぉあ}\par
っゎ		: \hinodeji[false]{っゎあ}\par
ゃゅょ	: \hinodeji[false]{ゃゅょあ}\par
ゔヵヶ	: \hinodeji[false]{ゔヵヶあ}\par

\begin{table}[tb]
	\caption{関数}
	\label{tab:functions}
	\centering

	\begin{tabular}{rl|c}
	\hline

	\hline
	\textbf{前} & \textbf{後} & \textbf{関数} \\
	\hline
		 漢字 & カタカナ & MecabYomi \\
		 漢字 & ひので字 & hinodeji \\
	\hline

	\hline
	\end{tabular}
\end{table}

\newpage

例文\par
\hinodeji{
青々と茂る森の中、小川のせせらぎが静かに響く。鳥たちのさえずりが風に乗り、陽光が木漏れ日となって地面を優しく照らしている。遠くの山々は霞み、季節の移ろいを穏やかに映し出す。道端には小さな花々あいうえおが咲き誇り、それぞれの色が調和を奏でている。時間の流れがゆるやかになり、心が自然と解きほぐされていくような気がした。

町の広場には賑やかな声が響き、人々が行き交っている。市場では新鮮な野菜や果物が並び、店主が元気な声で客を呼び込む。子どもたちは無邪気に駆け回り、笑い声が絶えない。通りには古い書店があり、並ぶ本の背表紙には長い歴史が刻まれている。ある一冊を手に取り、ページをめくると、過去の記憶が鮮明によみがえった。活字の並びに心を預け、物語の世界へと引き込まれていく。

夕暮れ時、空は茜色に染まり、街の灯りが次第にともり始める。波の音が遠くから聞こえ、穏やかな夜の訪れを告げている。ベンチに腰掛け、1日の出来事を振り返る。静かな時間の中で、心の奥深くにある思いがゆっくりと形を成していく。目を閉じると、風が優しく頬をなで、夜の帳が静かに降りてくる。すべてが調和し、ただそこにいることの心地よさを感じる瞬間だった。
}

\hinodeji{
コンピューターゲームのニューウェーブ。
チャッキーンとなるゲームをパッキーンと壊してく。
}
\end{document}
